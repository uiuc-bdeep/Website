\documentclass[11pt]{article}

\begin{abstract}

Building energy efficiency is a key component of climate and energy policy around the world.  The cost-effectiveness of retrofit programs depends upon the quality of pre-retrofit predictions that are used to allocate funds to projects.  This study develops an ML-based method for improving the prediction of net present benefits (NPB) in energy efficiency programs.  The study applies this method to retrofit data from 13 thousand homes in a large energy efficiency program in the United States. Pre-retrofit predictions from the ML-based model outperform the predictions made by the program's existing model.  If ML-based predictions were used to target investments to the top 75\% of homes in the sample, total benefits would exceed costs. Targeting investments to the top 40\% of homes increases the cost-effectiveness of energy efficiency investments by 18.8\%.

\end{abstract}
