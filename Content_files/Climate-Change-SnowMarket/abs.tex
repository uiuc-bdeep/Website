\documentclass[11pt]{article}

\begin{abstract}

Many mountain towns rely on climate amenities such as wintertime precipitation to generate local economic activity. However, climate models predict large reductions in annual snowfall that could greatly reduce tourism flows to these markets. Harnessing a unique panel of daily transactions from the short-term property rental market, we combine daily weather, daily resort snowpack, and daily resort snowfall to estimate the causal effect of changes in snowpack on visitation in 219 resort markets across the United States. We make three primary contributions to the study of climate change: 1) we develop a new method to estimate elasticities for climate amenities by matching the spatial and temporal variation in the level of the amenity with the frequency of related market transactions; 2) we derive state-specific snowpack elasticities for all major markets across the United States and find significant heterogeneity in the behavioral response across states; and 3) we estimate year-to-year variation in the recreation revenue from snowpack under current and future climate scenarios. We predict that resort markets could face reductions in local snow tourism of -40\% to -80\%, almost twice as large as previous estimates suggest. This translates to a lower-bound on the annual willingness to pay to avoid reductions in snowpack between \$1.55 billion (RCP4.5) and \$2.63 billion (RCP8.5) by the end of the century.

\end{abstract}
