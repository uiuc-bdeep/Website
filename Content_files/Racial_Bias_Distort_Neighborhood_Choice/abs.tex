\documentclass[11pt]{article}

\begin{abstract}

By constraining an individual's choice during a search, housing discrimination distorts sorting decisions away from true preferences and results in a ceteris paribus reduction in welfare. This study combines a large-scale  eld experiment with a residential sorting model to derive utility-theoretic measures of renter welfare loss associated with the constraints imposed by discrimination in the rental housing market. Results from experiments conducted in  ve cities show that key neighborhood amenities are associated with higher levels of discrimination. Results from the structural model indicate that discrimination imposes costs equivalent to 4.6% and 2.9% of the annual incomes for African American and Hispanic/LatinX renters, respectively. Search behavior results in greater welfare costs for African Americans as their incomes rise. Renters of color must make substantial investments in additional search to mitigate the costs of these constraints. We  nd that a naive model ignoring discrimination constraints yields signi cantly biased estimates of willingness to pay.

\end{abstract}
