\documentclass[11pt]{article}

\begin{abstract}

Optimal transportation policies depend on demand elasticities that interact across modes and vary across the population, but understanding how and why these elasticities vary has been an empirical challenge. Using an experiment with Uber in Egypt, we randomly assign large price discounts for transport services over a 3 month period to examine: (1) the demand for ride-hailing services and (2) the demand for total mobility (km/week). A 50% discount more than quadruples Uber usage and induces an increase of nearly 49% in total mobility. These effects are stronger for women, who are less mobile at baseline and perceive public transit as
unsafe. Technology-induced reductions in the price of ride-hailing services could generate substantial consumer surplus through combined mobility effects ($12 Billion PPP) but would be accompanied by considerable increases in external costs ($3.2 Billion PPP) resulting from increases in private vehicle travel.


\end{abstract}
