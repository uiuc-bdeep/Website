\documentclass[11pt]{article}

\begin{abstract}

By constraining an individual’s choice during a search, housing discrimination distorts sorting decisions away from true preferences and results in a ceteris paribus reduction in buyer welfare. This study combines a large-scale field experiment with a residential sorting model to derive utility-theoretic measures of renter welfare loss associated with the constraints imposed by discrimination in the rental housing market. In the five major metropolitan markets we study, we find that key amenities such as lower air toxicity and better schools are associated with higher levels of discrimination. We estimate welfare costs to renters from discriminatory constraints during the first phase of a search in these cities to be equivalent to 4.6% and 4.7% of the annual incomes for the average African American and Hispanic/LatinX households, respectively. For African American renters, these costs increase substantially at higher levels of income due to systematic exclusion from high amenity neighborhoods. African American renters face damages greater than 7% of income at income levels above $100,000 per year. We then study the effects of discrimination on revealed preferences for urban amenities. We find that discrimination drives a wedge between true preferences for key neighborhood amenities and those estimated without accounting for discriminatory constraints. A naive model that ignores those constraints understates the marginal willingness to pay of renters of color for a variety of neighborhood amenities by 2.2% to 5.2% relative to White renters.

\end{abstract}
