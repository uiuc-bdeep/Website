\begin{abstract}

In April 2014, the city of Flint, Michigan switched its drinking water supply from the Detroit water system to the Flint River as a temporary means to ease the city's financial troubles. This change in water supply unknowingly caused a major health crisis in the city. Over the course of a year and a half, it was revealed that the switch exposed Flint residents to dangerously high levels of lead, culminating in a public health emergency declaration in October 2015. In this paper, we use the events surrounding the Flint water crisis to explore how averting behaviors and local housing markets respond as a major environmental crisis unfolds. Home prices have declined significantly since the crisis was revealed. We find no effect of the crisis on the number of homes sold. We find that averting behaviors responded much more quickly to the change in drinking water quality. Purchases of bottled water increased substantially in the period after the switch and prior to the first emergency declaration. We also find an increase in sales of water filters, but only after the October 2015 declaration. We find no evidence of other averting or mitigating behaviors such as the purchases of carbonated beverages, alcohol, or over the counter pain medications and ointments.

\end{abstract}