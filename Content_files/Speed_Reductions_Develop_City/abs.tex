\documentclass[11pt]{article}

\begin{abstract}

Cities throughout the world are experimenting with more stringent speed limits in an effort to reduce road accidents. The effectiveness of these policies is of particular interest for developing world cities, where a disproportionate share of accident damages occur but also where extant congestion creates heightened concern about speed reduction.  This paper empirically evaluates a set of policies that changed traffic speed limits and enhanced its enforcement in São Paulo, Brazil.  We exploit the temporal and spatial heterogeneity of policy adoption to examine the effectiveness and impacts of enforcement using measurements of traffic accidents, traffic tickets issued by monitoring cameras, and a panel of repeat observations of real-time trip duration for a representative sample of travelers before and after the new regulations.  Our results indicate that speed limit reductions reduced accidents by 28.9% on treated road segments while not affecting traffic volume.  We find that camera-based enforcement augmented the effect of the speed limit reduction. We find that speed limit increases on major urban highways reduced trip times by 7.5% during off-peak periods, though we find no significant effects during peak hours.  We estimate that the social benefits from the reduction in road accidents are at least 2.34 times larger than the costs of longer commutes.  The benefits of accident reductions accrue largely to lower income pedestrians and motorcyclists, indicating that speed limit reductions have an important progressive welfare impact in developing country cities.

\end{abstract}
