\documentclass[11pt]{article}

\begin{abstract}

Our results indicate that improving safety near parks can turn them from public bads to goods. Ignoring complementarities may lead to i) undervaluing the potential value of public goods; ii) overestimating heterogeneity in preferences; and iii) understating the value of public goods to low income households. Recent reductions in crime have “unlocked” $3 billion in property value in these three cities. Still over half of the potential value of park proximity (approximately $9 billion) remains locked in.

\end{abstract}