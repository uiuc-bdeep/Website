%%%%%%%%%%%%%%%%%%%%%%%%%%%%%%%%%%%%%%%%%
% Medium Length Professional CV
% LaTeX Template
% Version 2.0 (8/5/13)
%
% This template has been downloaded from:
% http://www.LaTeXTemplates.com
%
% Original author:
% Trey Hunner (http://www.treyhunner.com/)
%
% Important note:
% This template requires the resume.cls file to be in the same directory as the
% .tex file. The resume.cls file provides the resume style used for structuring the
% document.
%
%%%%%%%%%%%%%%%%%%%%%%%%%%%%%%%%%%%%%%%%%

%----------------------------------------------------------------------------------------
%	PACKAGES AND OTHER DOCUMENT CONFIGURATIONS
%----------------------------------------------------------------------------------------

\documentclass{resume} % Use the custom resume.cls style

\usepackage[left=0.75in,top=0.6in,right=0.75in,bottom=0.6in]{geometry} % Document margins

\reversemarginpar
\newcommand{\MarginText}[1]{\leavevmode\marginpar{\raggedleft#1}\ignorespaces}
\usepackage{enumitem}

\name{Peter Christensen} % Your name
\address{431 Mumford Hall \\ 1301 W. Gregory Dr.} % Your address
\address{University of Illinois, Urbana-Champaign} % Your secondary addess (optional)
\address{\bf Office: 217-300-3054 \\ Email: pchrist@illinois.edu} % Your phone number and email

%----------------------------------------------------------------------------------------
%	PUBLICATIONS PREAMBLE
%----------------------------------------------------------------------------------------

% zzy: duplicate document class
% \documentclass[10pt,a4paper]{article}

%COMPILING: IMPORTANT INFORMATION
% You'll need to generate a .bbl file from
% the .bib file first, otherwise the publications
% will not show up in the cv.

% in TeXstudio, this means
% 1. F1 or F6 for usual compiling (generate .aux)
% 1. F8 for bibliography compiling (generate .bbl)
% 2. F1 or F6 for usual compiling once more(generate .pdf)

% use your favourite citation style here
\usepackage{apacite}

% zzy: duplication
% \usepackage[left=2.5cm,right=2.5cm,top=1.5cm,bottom=2.5cm]{geometry}


% as bibentry and hyperref clash in some cases, this is a workaround
% as suggested by the following two links:
%https://tex.stackexchange.com/questions/227933/using-bibentry-to-cite-in-text-reference-with-apacite
% and
%https://tex.stackexchange.com/questions/65348/clash-between-bibentry-and-hyperref-with-bibstyle-elsart-harv/65401#65401
\usepackage{bibentry}
\makeatletter\let\saved@bibitem\@bibitem\makeatother
\usepackage[colorlinks=true]{hyperref}
\makeatletter\let\@bibitem\saved@bibitem\makeatother


% hanging publications: 1st line starts at beginning of line, 
% further lines are placed a bit to the right
\usepackage{hanging}
\newcommand\publication[1]{%
	\smallskip\par\hangpara{1.5em}{1}\bibentry{#1}\smallskip
}

%-----------------------------------------------------------------------

\begin{document}
% bibliography


%----------------------------------------------------------------------------------------
%	EDUCATION SECTION
%----------------------------------------------------------------------------------------

\begin{rSection}{Education}


{\bf Ph.D. Environmental and Resource Economics, Yale University}  \hfill { 2015}  \\
{\bf M.E.Sc. Environmental Economics and Policy (concentration), Yale F\&ES}  \hfill {2009} \\
{\bf B.A. Psychology, Summa Cum Laude, University of California, Davis}  \hfill {2004} \\

\end{rSection}
%----------------------------------------------------------------------------------------
%	WORK EXPERIENCE SECTION
%----------------------------------------------------------------------------------------
\begin{rSection}{Academic Appointments}
{\bf Assistant Professor}  \hfill { 2015-}  \\
{Department of Agricultural and Consumer Economics} \\
{University of Illinois, Urbana-Champaign} \\
{\bf Core Faculty Member}  \hfill { 2016-} \\
{National Center for Supercomputing Applications} \\
{University of Illinois, Urbana-Champaign}\\
\end{rSection}
%%-------------------------------------------------------------------------
\begin{rSection}{Publications}
\MarginText{\nobibliography{publications.bib}}
\publication{gillingham2018forthcoming}
\publication{christensen2018uncertainty}
\publication{christensen2016geographic}
\publication{mendelsohn2010ricardian}
\end{rSection}

\begin{rSection}{Working Papers and Papers in Review}	
%%%
\publication{christensen2018sorting}
%(\href{https://arxiv.org/abs/1706.06969}{NBER Working Paper No. 24826}, \href{https://arxiv.org/pdf/1706.06969.pdf}{pdf}, \href{https://github.com/rgeirhos/object-recognition}{data and materials})\\
%%%
\publication{albouyunlocking}
\publication{christensensuburbs}
%%%
\end{rSection}

%%------------------------------------------------------------------------

\begin{rSection}{Work in Progress}
%%%
\publication{flintwater}
\publication{welfarediscrimination}
\publication{christensenspeedlimits}
%%%
\end{rSection}

%----------------------------------------------------------------------------------------
%	TECHNICAL STRENGTHS SECTION
%----------------------------------------------------------------------------------------

%\begin{rSection}{Technical Strengths}

%\begin{tabular}{ @{} >{\bfseries}l @{\hspace{6ex}} l }
%Computer Languages & Prolog, Haskell, AWK, Erlang, Scheme, ML %\\
%Protocols \& APIs & XML, JSON, SOAP, REST \\
%Databases & MySQL, PostgreSQL, Microsoft SQL \\
%Tools & SVN, Vim, Emacs
%\end{tabular}

%\end{rSection}


%----------------------------------------------------------------------------------------
%	EXAMPLE SECTION
%----------------------------------------------------------------------------------------

%\begin{rSection}{Section Name}

%Section content\ldots

%\end{rSection}

%----------------------------------------------------------------------------------------

\bibliographystyle{apalike}
\end{document}
