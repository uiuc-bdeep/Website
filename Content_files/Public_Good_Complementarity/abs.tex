\documentclass[11pt]{article}

\begin{abstract}

Public-good complementarities have important implications for economic valuation, but are understudied. We find that public safety and urban parks are powerful complements using detailed crime and housing data in Chicago, New York, and Philadelphia. Ignoring complementarities leads to i) undervaluing public goods; ii) inefficient investments in public capital in high-crime areas; iii) the (wrong) conclusion that public goods are a luxury; iv) overestimation of preference heterogeneity. Our results indicate that reducing crime near parks can turn them from public "bads" to goods. Reductions over the past two decades has ""unlocked" $2.8 billion in taxable property value in our sample cities and has the potential to unlock another $8 billion.

\end{abstract}