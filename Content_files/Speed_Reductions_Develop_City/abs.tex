\documentclass[11pt]{article}

\begin{abstract}

Cities are experimenting with more stringent speed limits in an effort to reduce road accidents, which are the leading cause of unnatural deaths worldwide. The effectiveness of these policies is of particular interest in developing countries, where a disproportionate share of accident damages occur but also where congestion creates heightened concern about speed regulations.  We evaluate a speed limit reduction program in São Paulo, Brazil using a dynamic event study design and measurements of 125 thousand traffic accidents, 38 million traffic tickets issued by monitoring cameras, and 1.4 million repeat observations of real-time trip durations before and after a regulatory change. We estimate that the program reduced accidents by 21.7% on treated roads and resulted in 1,889 averted accidents within the first 18 months, with larger effects on roads with camera-based enforcement.  The program also affected travel times on treated roads (5.5%), though the social benefits from reduced accidents are at least 1.77 times larger than the social costs of longer trip times.  The benefits of accident reductions accrue largely to lower income pedestrians and motorcyclists, indicating that speed limit reductions may have important impacts on low income residents in developing country cities.

\end{abstract}
