\documentclass[11pt]{article}

\begin{abstract}

Public goods may exhibit complementarities that are essential for determining their individual value. Our results indicate that improving safety near parks can turn them from public bads to goods. Ignoring complementarities may lead to i) undervaluing the potential value of public goods; ii) overestimating heterogeneity in preferences; and iii) understating the value of public goods to minority households.  Recent reductions in crime have "unlocked" $5.5 billion in property value in Chicago, New York and Philadelphia. Still over half of the potential value of park proximity (over $10.5 billion) remains locked in.

\end{abstract}
