\documentclass[11pt]{article}

\begin{abstract}

Residents of developing country cities may be mobility-constrained for a variety of reasons, including low levels of access to reliable transit infrastructure, concerns about personal safety and harassment risk, and barriers associated with private transport. Using an experiment with Uber in Egypt, we provide individuals with large subsidies over a 3 month period for Mobility-on-Demand (MoD) services in order to examine the demand response on (1) Uber usage and (2) total travel per week.  We find strong responses on both outcomes to fare reductions.  For the average participant in our study, a 25\% discount more than doubles Uber usage and induces an increase of 13\% in total travel while a 50\% discount more than quadruples Uber usage and induces an increase of 42\% in total travel relative to control. These effects are stronger for women, who are less mobile at baseline and report low levels of personal safety on bus, metro, and taxi trips.  We find that price reductions on MoD services induce substitution away from buses and taxis, and increase women's experienced safety in recent travel. Using information on impacts on mode use, we estimate that the price elasticity for private vehicle kilometers traveled is -1.28, implying that as prices fall, demand for private versus mass transit modes will increase substantially.  We examine implications for congestion and other external costs.

\end{abstract}
